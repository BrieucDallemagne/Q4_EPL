\section{Vecteurs}
\subsection{Base orthonormée}
On dit qu'une base $(e_1, e_2, e_3)$ est orthonormée si
\begin{equation*}
    \textbf{e}_i \cdot \textbf{e}_j = \delta_{ij}
\end{equation*}
Où $\delta_{ij}$ est le symbole de Kronencker. Il est défini comme suit

\begin{equation*}
    \delta_{ij} = 
    \begin{cases}
        1 & \text{ si $i = j$}\\
        0 & \text{ si $i \neq j$}\\
    \end{cases}
\end{equation*}\\
Dans ce cours, uniquement les bases orthonormées seront considérées.

\subsection{Notation d'Einstein}

Afin de simplifier les notations, nous utiliserons dans ce cours la notation de Einstein. Celle-ci permet de simplifier l'écriture des sommes.\\

Lorsqu'on est face à une expression faisant intervenir des indices, il faut distinguer les indices \textbf{répétés} et les indices \textbf{muets}.\\
\\
Prenons l'expression suivantes :
\begin{equation*}
    u_j = \alpha v_j
\end{equation*}
Ici, l'indice $j$ est considéré comme muet. Il intervient \textbf{des deux côtés de l'égalité}\\
\\
En revanche, si on écrit l'égalité suivante :
\begin{align*}
    \Vec{\mathbf{u}} &= \sum_i u_i \mathbf{\hat{e}_i}\\
    \vec{\mathbf{u}} &= u_i \mathbf{\hat{e}_i}
\end{align*}
L'indice $i$ est dit répété. En effet, il n'apparaît que d'un côté de l'égalité. Cette notation $\vec{\mathbf{u}} &= u_i \mathbf{$\hat{e}_i$}$ sous-entend en fait la somme pour $i = 1, 2, 3$


\subsection{Changement de base}

Soit une base $(\textbf{e}_i)$ et une base $(\textbf{e}_i')$ et soit $\textbf{P} \in \mathbb{R}^{3\times 3}$ la matrice de changement de base.\\

On a 
\begin{equation*}
    \textbf{e}_i' = P_{ij}e_j
\end{equation*}



\section{Tenseurs}

Un tenseur est un objet mathématique, étendant les concepts de scalaire et de vecteur, qui peut être représenté par ses composantes dans une base donnée. Cet objet se doit également d'obéir à des règles précises de transformation lors d'un changement de base.
\subsection{Pourquoi utiliser des tenseurs ?}
Pour caractériser les certaines propriétés telles que la déformation ou les contraintes, nous avons besoin de connaître 6 informations pour un point de l'espace (trop pour un vecteur). Il est donc nécessaire de généraliser le concept de vecteur et arriver aux tenseurs.

\subsection{Représentation (notation) des tenseurs}
Tout d'abord, il faut toujours garder en tête qu'\textbf{un tenseur est exprimé dans une base}\\

Un tenseur peut s'exprimer de plusieurs façons différentes : 
\begin{itemize}
    \item De manière indicielle : $\mathbf{\undertilde{a}} = (a_{ij}) \text{ in } (\mathbf{\hat{e}_i})$
    \item De manière matricielle :  $\mathbf{\undertilde{a}} = [a] \text{ in } (\mathbf{\hat{e}_i})$
\end{itemize}


\subsection{Opérations sur les tenseurs}
\begin{itemize}
    \item Le produit tensorielle de deux vecteurs : 
    \begin{equation*}
        \mathbf{\undertilde{a}} = \vec{\mathbf{u}} \otimes \vec{\mathbf{v}} \text{ où, } a_{ij} = u_i v_j \text{ in } (\mathbf{\hat{e}_i})
    \end{equation*}
    Le produit tensoriel peut s'exprimer de manière matricielle comme suit 
    \begin{equation*}
        \mathbf{\undertilde{a}} = \{\mathbf{u}\} \{\mathbf{v}\} ^T
    \end{equation*}
    Où $\mathbf{u}$ et $\mathbf{v}$ sont deux vecteurs colonnes.\\

    \textbf{Conséquence} : N'importe quel tenseur $\mathbf{\undertilde{a}}$ peut se décomposer comme suit : 
    \begin{equation*}
    \mathbf{\undertilde{T}} = a_{ij} \mathbf{\hat{e}_i} \otimes \mathbf{\hat{e}_j}
\end{equation*}

\item Le produit entre tenseur et vecteur : 
    \begin{equation*}
        \Vec{\mathbf{v}} = \mathbf{\undertilde{a}}(\Vec{\mathbf{u}}) = \mathbf{\undertilde{a}} \cdot \Vec{\mathbf{u}} \text{ où, } v_i = a_{ij} v_j
    \end{equation*}

\item La trace d'un tenseur : 
    \begin{equation*}
        tr(\mathbf{\undertilde{a}}) = a_{ii}
    \end{equation*}

\item Le produit contracté sur un indice :\\

    Entre un tenseur $\mathbf{\undertilde{a}}$ et un vecteur $\Vec{\mathbf{v}}$ :
    \begin{equation*}
        (\mathbf{\undertilde{a}} \cdot \Vec{\mathbf{v}})_i = a_{ij}v_j
    \end{equation*}

    Entre un tenseur $\mathbf{\undertilde{a}}$ et un tenseur $\mathbf{\undertilde{b}}$ : 

    \begin{equation*}
        (\mathbf{\undertilde{a}} \cdot \mathbf{\undertilde{b}})_{ij} = a_{ik}b_{kj}
    \end{equation*}

\item Le produit contracté sur deux induces : 
    \begin{equation*}
        \mathbf{\undertilde{a}} : \mathbf{\undertilde{b}} = a_{ij}b_{ji} = tr(\mathbf{\undertilde{a}} \cdot \mathbf{\undertilde{b}})
    \end{equation*}
\end{itemize}


\subsection{Les tenseurs symétriques}

Un tenseur symétrique $\mathbf{\undertilde{a}}$ peut être décomposé en une composante déviatorique et une composante sphérique.


\begin{equation*}
    \mathbf{\undertilde{a}} = \underbrace{\mathbf{\undertilde{a}} - \frac{1}{3} tr(\mathbf{\undertilde{a}}) \mathbf{\undertilde{1}}}_{deviatoriµque} + \underbrace{\frac{1}{3} tr(\mathbf{\undertilde{a}})\mathbf{\undertilde{1}}}_{spherique}
\end{equation*}

Cette décomposition sera utile plus tard pour représenter respectivement des changement de forme et des changements de volume.

\subsection{Les invariants}

Les invariants sont des propriétés d'un tenseur qui \textbf{ne dépendent pas} de la base dans laquelle il est exprimé.\\

\textit{Pour vérifier si on n'a pas fait d'erreur dans un changement de base, il suffit de vérifier que les invariants sont conservés.}\\

Les invariants d'un tenseur sont : 
\begin{itemize}
    \item La trace 
    \item Le déterminant
\end{itemize}